\input{"C:/Users/spileggi/Google Drive/STAT 330/Labs/LabStyle.tex"}

\begin{document}
\hd{17}
\labn{17}
\vskip10pt

\noindent Use the 2012 Olympic Medalists data set \ttt{O2012.sas7bdat} for the following exercises (this data set name starts with an ``oh'' and not a zero).  \textbf{\emph{Example} output is provided on the last pages.}
\begin{enumerate}
\item Create a SAS library to access the \ttt{O2012.sas7bdat}.
\item While for certain sports athletes peak at younger ages, in some sports there is more variability in the age of athletes.  Using \ttt{PROC SQL}, create a summary of the olympics data by sport that
\begin{enumerate}
\item displays the sport name
\item displays the number of medalists that participated in each sport
\item displays the standard deviation of athlete's age for each sport
\item is sorted by standard deviation in descending order
\end{enumerate}
Which sport has the largest variability in age?  Note your findings in a comment in your SAS code.
\item Suppose all Olympic medalists of the same country are traveling together and need to get on an elevator that has a 2000kg weight limit.  Which countries could fit all of their medalists on this elevator?  Using \ttt{PROC SQL}, create a summary of the olympics data by country that
\begin{enumerate}
\item displays the country name
\item displays the number of medalists for each country
\item sums the total weight of all medalists per country
\item creates a new variable called \ttt{elevator} which has a value of ``yes'' if the total weight is less than or equal to 2000, and ``no'' otherwise; make sure that countries with a missing value for total weight also have a missing value for \ttt{elevator}
\item is sorted by total weight in descending order
\end{enumerate}
In examining the output, how many countries wouldn't be able to put all of their medalists on the elevator because of the weight limit?  Note your findings in a comment in your SAS code.
\item Using \ttt{PROC SQL}, re-create the table from the previous two labs where you summarized the olympics data by country.  Display the following variables:
\begin{enumerate}
\item country
\item number of medalists in each country 
\item total number of medals won in each country 
\item average number of medals won per athlete in each country
\item maximum number of medals won per athlete in each country
\item average age of medalists in each country
\item average weight of medalists in each country
\item percent of medalists that are female in each country
\item percent of medalists that are male in each country
\end{enumerate}
\item[]
\emph{Hint:} For the last two variables regarding the percent of males and the percent of females, use a \ttt{CASE} statement within a \ttt{SUM} expression such as
\item[] \fbox{\ttt{SUM(CASE WHEN \emph{condition} THEN \emph{value1} ELSE \emph{value2} END) AS \emph{newvariable} }}
\end{enumerate}
\newpage
\includegraphics[trim={0cm 0cm 0cm 0cm},clip,width=1.0\textwidth]{Q2.pdf}
\includegraphics[trim={0cm 0cm 0cm 0cm},clip,width=1.0\textwidth]{Q3.pdf}
\includegraphics[trim={0cm 0cm 0cm 0cm},clip,width=1.0\textwidth]{Q4.pdf}

\end{document} 