
%\documentclass[12pt]{report}

\documentclass[12pt]{article}
%\usepackage{natbib}  % used for citations
\usepackage[parfill]{parskip} %used for formatting style of text



\usepackage{graphicx,fancyhdr}
\usepackage{amssymb,amsmath}
\usepackage{epigraph,fancyvrb,eqparbox}
\usepackage[multiple]{footmisc}
\usepackage{menukeys}
\usepackage{menukeys}
\usepackage{url}
\usepackage[colorlinks = true, linkcolor = blue, urlcolor = blue]{hyperref}
\usepackage{setspace}

\pagestyle{fancyplain}

%\usepackage{hyperref}
%\usepackage{epsf,psfig,graphicx,fancyheadings}
% \textwidth 7in
% \textheight 9in
% \oddsidemargin 0in
% \topmargin -.25in

%-----------------------------------------------
% The following settings are from Dr. Davidian's
% ST810A Handout on Advanced LaTeX Features

%\setlength{\paperheight}{11.0in}
%\setlength{\paperwidth}{8.5in}

%%%%%%%%%%%%%%%%%%%%%%%%%%%%%%%%%%%%%%%%%%%%%%%%%
% For Desktop @ CalPoly (for Postscript)

%\setlength{\oddsidemargin}{0.5in}
%\setlength{\evensidemargin}{0.5in}
%\setlength{\topmargin}{-.5in}

%%%%%%%%%%%%%%%%%%%%%%%%%%%%%%%%%%%%%%%%%%%%%%%%%
% For Laptop @ Calpoly (for Postscript)

% \setlength{\oddsidemargin}{0.in}
% \setlength{\evensidemargin}{0.in}
% \setlength{\topmargin}{0.25in}

%%%%%%%%%%%%%%%%%%%%%%%%%%%%%%%%%%%%%%%%%%%%%%%%%
% For Desktop @ CalPoly (for PDF)

%\setlength{\oddsidemargin}{0.in}
%\setlength{\evensidemargin}{0.in}
%\setlength{\topmargin}{-.5in}
%
%%%%%%%%%%%%%%%%%%%%%%%%%%%%%%%%%%%%%%%%%%%%%%%%%%
%% For Laptop @ Calpoly (for PDF)
%
%% \setlength{\oddsidemargin}{0.in}
%% \setlength{\evensidemargin}{0.in}
%% \setlength{\topmargin}{0.25in}
%
%
%
%\setlength{\oddsidemargin}{0.0in}
%\setlength{\topmargin}{-0.5in}
%\setlength{\headheight}{0.20in}
%\setlength{\headsep}{3ex}
%\setlength{\baselineskip}{2ex}
%\setlength{\textheight}{9in}
%\setlength{\textwidth}{6.4in}
%\renewcommand{\baselinestretch}{1.1}

% Sets margins to 1 in
\addtolength{\oddsidemargin}{-.5in}%
\addtolength{\evensidemargin}{-.5in}%
\addtolength{\textwidth}{1in}%
\addtolength{\textheight}{1.3in}%
\addtolength{\topmargin}{-.8in}%

%\setlength{\headheight}{0.20in}
%\setlength{\headsep}{3ex}
%\setlength{\headrulewidth}{0.2pt}
%\setlength{\footrulewidth}{0.15pt}
%\setlength{\parskip}{2.3ex}
% %set to no indentation
%\setlength{\parindent}{0.0in}
%\setlength{\baselineskip}{2ex}
%\setlength{\textheight}{9.in}
%\setlength{\textwidth}{6.5in}

\def \doublespace{\openup 2\jot}
% For double or 1.5 spacing
%\renewcommand{\baselinestretch}{1.5}
\tolerance=500

\def\boxit#1{\vbox{\hrule\hbox{\vrule\kern6pt
\vbox{\kern6pt#1\kern6pt}\kern6pt\vrule}\hrule}}
\renewcommand{\theequation}{\thesection.\arabic{equation}}
% The following for TOC
%\renewcommand{\thepage}{\roman{page}}
% to be followed by this for the main text
\renewcommand{\thepage}{\arabic{page}}


%-----------------------------------------------

%%%%%%%%%%%%%%%%%%%%%%%%%%%%%%%%%%%%%%
%Define any shortcut aliases below

\newtheorem{theo}{Theorem}[section]

\newenvironment{note}{\begin{quote}\emph{Note:\ }}{\end{quote}}
\newenvironment{defn}{
\begin{description}
\item[Definition ]}
{\end{description}}

\newenvironment{ttscript}[1]{%
    \begin{list}{}{%
    \settowidth{\labelwidth}{\texttt{#1}}
    \setlength{\leftmargin}{\labelwidth}
    \addtolength{\leftmargin}{\labelsep}
    \setlength{\parsep}{0.5ex plus0.2ex minus0.2ex}
    \setlength{\itemsep}{0.3ex}
    \renewcommand{\makelabel}[1]{\texttt{##1\hfill}}}}
    {\end{list}}

\newcommand{\bt}{\begin{tabular}}
\newcommand{\et}{\end{tabular}}
\newcommand{\bc}{\begin{center}}
\newcommand{\ec}{\end{center}}
\newcommand{\bi}{\begin{itemize}}
\newcommand{\ei}{\end{itemize}}
\newcommand{\be}{\begin{enumerate}}
\newcommand{\ee}{\end{enumerate}}
\newcommand{\bq}{\begin{quote}}
\newcommand{\eq}{\end{quote}}
\newcommand{\vect}[1]{\mbox{\boldmath $ #1$}}
\newcommand{\avg}[1]{$\overline{#1}$}
\newcommand{\bmp}{\begin{minipage}}
\newcommand{\emp}{\end{minipage}}
\newcommand{\hr}{\u{\hspace{7in}}}
\newcommand{\sr}{\u{\hspace{5in}}}
\newcommand{\chs}{\chi^2}

\newcommand{\labn}[1]{\Large{\textbf{\fbox{Lab #1}}}\hspace{0.1in} \normalsize{\emph{Some of these problems may be more challenging than others. Please feel free to work with others, attend office hours, or post on the course discussion forum if you need help.  While collaboration with other students is encouraged, each student is responsible for submitting his or her own work.  This assignment should be submitted in one well-commented SAS program.  For any questions that require a written answer, do so in the SAS comments.  Be sure to re-name the uploaded SAS scripts according to the naming convention}} \texttt{LastnameFirstinitial\textunderscore Lab\#.sas} (\emph{e.g.,} \texttt{PileggiS\textunderscore Lab#1.sas}).}


\newcommand{\hd}[1]{\lhead{STAT 330/530: Lab #1}\rhead{Pileggi, FA17}}
\newcommand{\bs}{\underline{\hspace{0.5in}}}

%\newcommand{\bv}{\footnotesize
%\bmp{.5\textwidth}
%\begin{Verbatim}[frame=single,label=SAS Code,commandchars=\\\{\}],xrightmargin=.5\textwidth}
%
%\newcommand{\ev}{\end{Verbatim}
%\emp
%\normalsize}

\newcommand{\bv}{\begin{code}}
\newcommand{\ev}{\end{code}}

 \newenvironment{code}[1]%
  {\vspace{.1in}\footnotesize\Verbatim[frame=single,label=SAS Code,commandchars=\\\{\},xrightmargin=#1\textwidth,framesep=.2in,labelposition=all]}
  {\endVerbatim\normalsize}

\newenvironment{craw}[2]%
{\vspace{.1in}\footnotesize\Verbatim[frame=single,label=#2,commandchars=\\\{\},xrightmargin=#1\textwidth,framesep=.2in,labelposition=all]}
  {\endVerbatim\normalsize}

\newenvironment{cbox}[1]%
{\vspace{.1in}\footnotesize\Verbatim[frame=single,commandchars=\\\{\},xrightmargin=#1\textwidth,framesep=.2in,labelposition=all]}
  {\endVerbatim\normalsize}

\newcommand{\head}[1]{\large \textbf{#1} \normalsize}

\newcommand{\ttt}[1]{\textbf{\texttt{#1}}}


\newcommand{\bsval}[1]{\underline{\hspace{0.2in}{[#1]}\hspace{0.2in}}}

\newcommand{\ttb}{\textbf}
\newcommand{\tte}{\emph}
\newcommand{\ttu}{\underline}



\newcommand{\jdhr}{\vspace{0.2in}\hrule}


\newcommand{\uspace}[1]{\underline{\hspace{#1}}}

\newenvironment{ident}{\begin{list}{}{}
         \item[]}{\end{list}}

\newenvironment{proposition}{
\begin{description}
\item[Proposition: ]}
{\end{description}}

\newcommand{\bpr}{\begin{proposition}}
\newcommand{\epr}{\end{proposition}}



% \newenvironment{example}
%     {
%         \begin{list}{\textbf{Example:}}
%         {
%         \settowidth{\labelwidth}{}
%         \setlength{\leftmargin}{\labelwidth}
%         }
%     }
%     {\end{list}}


\newenvironment{example}{
\jdhr \vspace{-.17in}\jdhr
\textbf{Example: }}
{}

\newcommand{\bex}{\begin{example}}
\newcommand{\eex}{\end{example}}

\newenvironment{onyourown}{
\jdhr \vspace{-.17in}\jdhr
\textbf{On Your Own: }}
{}

\newcommand{\boy}{\begin{onyourown}}
\newcommand{\eoy}{\end{onyourown}}


%\newenvironment{debug}{
%\jdhr \vspace{-.17in}\jdhr
%\ttb{Debug the Code}
%\fbox{
%\bmp{.95in}
%\includegraphics[height=.35in]{C:/images/bug4.jpg}\includegraphics[height=.35in]{C:/images/buggy8.jpg}
%\emp}
%}
%{\jdhr}

\newenvironment{debug}{
\jdhr \vspace{-.17in}\jdhr
\ttb{Debug the Code: }
\fbox{
\bmp{.95in}
\includegraphics[height=.35in]{C:/images/bug4.jpg}\includegraphics[height=.35in]{C:/images/mushi90.jpg}
\emp}
}
{}


\newcommand{\bbug}{\begin{debug}}
\newcommand{\ebug}{\end{debug}}


\begingroup
  \catcode `_=11
  \gdef\myuscore{_}
  \catcode `~=11
  \gdef\mytilde{~}
  \catcode `\|=0
  \catcode `\\=11
  |gdef|mybs{\}
|endgroup

%Define any shortcut aliases above


%....................................................................
%....................................................................
%....................................................................
%....................................................................
%....................................................................
%....................................................................
%....................................................................
%....................................................................



\usepackage{amssymb}
				