
\input{"C:/Users/spileggi/Google Drive/STAT 330/Lectures/SlideStyle.tex"}



\title[Lecture 5]{SAS Macros}
\author[Pileggi]{Shannon Pileggi}

\institute[STAT 330]{STAT 330}

\date{}


\begin{document}

\begin{frame}
\titlepage
\end{frame}

\begin{frame}
\frametitle{OUTLINE\qquad\qquad\qquad} \tableofcontents[hideallsubsections]
\end{frame}



%===========================================================================================================================
\section[Getting started]{Getting started}
%===========================================================================================================================
\subsection{}
%\begin{frame}
%\tableofcontents[currentsection, hideallsubsections]
%\end{frame}



\begin{frame}
\ft{SAS macros}
\bi
\item The SAS macro facility is a \emph{text processing facility}
\item It allows us to insert/include line(s) of SAS code anywhere in the entire program
\item This provides a \ttb{very} convenient way to automate many processes
\item It is very much like having a handy recording of SAS code which you can play back whenever you need
\item Macros can be broken down into two main types: \\
\bi
\item single variable insertion \\
\item multiple lines insertion
\ei
\ei
\end{frame}


\begin{frame}
\ft{The Macro Processor and the Standard SAS Processor}
\bi
\item In the presence of macro code, SAS will go through an initial scan of your code and `resolve' any macros first.
\item After the initial scan, the appropriate line(s) of SAS code are `inserted' (it's like a program that writes a program)
\item Finally, SAS compiles the full code and executes as usual
\ei
\end{frame}

\begin{frame}
\ft{Macro triggers}
When a SAS program is submitted, two token sequences are recognized as \emph{macro triggers}: \\
\vskip15pt
\begin{enumerate}
    \item \ttt{\&name-token} - a macro variable reference
    \item \ttt{\%name-token} - a macro statement, function, or call
\end{enumerate}
\vskip15pt
(A \emph{token} is a fundamental unit of text.)
\end{frame}


\begin{frame}
\ft{Debugging macros}
Items that are underlined represent the default SAS settings:
\bi
\small{
\item \ttt{\textbf{\underline{MERROR}} | NOMERROR} -- issues a warning in the log window when attempting to invoke a macro that does not exist.
\item \ttt{\textbf{\underline{SERROR}} | NOSERROR} -- issues a warning in the log window when attempting to use a macro variable that does not exist.
\item \ttt{\textbf{MLOGIC | \underline{NOMLOGIC}}} -- prints (in the log window) details of every macro step.
\item \ttt{\textbf{MPRINT | \underline{NOMPRINT}}} -- prints (in the log window) details of what SAS ultimately ``sees'' during the Standard SAS Processor stage.
\item \ttt{\textbf{SYMBOLGEN | \underline{NOSYMBOLGEN}}} -- prints (in the log window) the resolved values of any macro variables. }
\ei
Use the following to ensure ALL your macro tools are made available to you:
\hspace{1.5in}\fbox{\ttt{OPTIONS MPRINT MLOGIC SYMBOLGEN;}}
\end{frame}


%===========================================================================================================================
\section[Single variable insertion]{Single variable insertion}
%===========================================================================================================================
\subsection{}
\begin{frame}
\tableofcontents[currentsection, hideallsubsections]
\end{frame}

\begin{frame}
\ft{Macro Variable: The Single Variable Insertion}
\bi
\item Macro variables have a \emph{single} value and do \emph{not} belong to a data set
\item When reference, macro variable names are prefixed with an ampersand (\ttt{\&})
\item All macro variables are stored as \emph{character} based variables
\item You may name a macro variable whatever you wish, but do not use \ttt{sys} as the first three letters of a macro variable. Such variables are reserved for special purposes.
\item A macro variable may have a \ttb{global scope} (can be used any where in the code) or a \ttb{local scope} (used only in a macro).
\ei
\end{frame}

\begin{frame}[fragile]
\ft{Automatic macro variables}
\bi
\item SAS has automatic macro variables that begin with the prefix \ttt{sys}
\item \url{http://support.sas.com/documentation/cdl/en/mcrolref/61885/HTML/default/viewer.htm#a003167023.htm}
\ei
\bmp{1.0\textwidth}
\footnotesize
\begin{code}{.0}
TITLE "Contents of Baseball Data on  \textcolor{OrangeRed}{&sysdate9}" ;
PROC CONTENTS DATA = sashelp.baseball VARNUM ;
RUN ;
TITLE ;
\end{code}
\emp
\end{frame}



\begin{frame}
\ft{Macro Variable: The Single Variable Insertion}
\bi
\item  To create a basic macro variable we use \fbox{\ttt{\%LET macro\textunderscore variable\textunderscore  name = \emph{value} ;}}
\item \ttt{\%LET} statements are valid in open code (any where in SAS program)
\item When \emph{assigning} a macro variable a value
\bi
\item do not do \ttt{\%LET \textcolor{OrangeRed}{\&}macro\textunderscore variable\textunderscore = \emph{value} ;}
\item do not put quotes around the \ttt{\emph{value}}
\ei
\item This is useful for changing values during a SAS program without having to change the entire program itself
\item  To use the macro variable that you've created call it with \ttt{\&macro\textunderscore variable\textunderscore  name}
\ei
\end{frame}

\begin{frame}[fragile]
\ft{Resolving macro variables}
\begin{center}
\bmp{0.6\textwidth}
\footnotesize
\begin{code}{.0}
\%LET my\_GPA = 3.3;
\%LET country = New Zealand;
\end{code}
\emp
\end{center}
\vskip10pt
\begin{tabular}{ll}
\ttb{SAS Code} &  \fbox{\ttt{IF GPA = \textcolor{OrangeRed}{\&my\_GPA};}}\\
\ttb{Resolves to}  & \hspace{0.02in}  \ttt{IF GPA = \textcolor{OrangeRed}{3.3};}\\
[3ex]
\ttb{SAS Code} & \fbox{\ttt{title "Addresses in \textcolor{OrangeRed}{\&country}";}} \\
\ttb{Resolves to}  &   \ttt{title "Addresses in \textcolor{OrangeRed}{New Zealand}";}
\end{tabular}
\end{frame}



\begin{frame}[fragile]
\ft{Baseball data}
\bmp{1.0\textwidth}
\footnotesize
\begin{code}{.0}
TITLE "Data = \textcolor{OrangeRed}{sashelp.basesball}, Obs = \textcolor{OrangeRed}{10}" ;
PROC PRINT DATA = \textcolor{OrangeRed}{sashelp.baseball} (OBS = \textcolor{OrangeRed}{10});
RUN ;
TITLE ;
\end{code}
\emp
\vskip15pt
\oyo Convert the dataset name of \textcolor{OrangeRed}{\ttt{sashelp.basesball}} and the number of observations printed \textcolor{OrangeRed}{\ttt{10}} to macro variables named \textcolor{OrangeRed}{\ttt{dsn}} and \textcolor{OrangeRed}{\ttt{num}}.
\end{frame}

\begin{frame}[fragile]
\bmp{0.5\textwidth}
\footnotesize
\begin{code}{.0}
\%let x=15;
\%let y=10;
\%let z=\&x-\&y;
\end{code}
\emp
\vskip10pt
\begin{clicker}{What is the value of the SAS macro variable \ttt{z}?}
\begin{enumerate}
    \item \ttt{5}
    \item \ttt{15-10}
    \item \ttt{x-y}
    \item \ttt{\&15-\&10}
    \item \ttt{error}
\end{enumerate}
\end{clicker}
\end{frame}


\begin{frame}[fragile]
\ft{Macro variables for path names}

\bmp{0.80\textwidth}
\footnotesize
\begin{craw}{.0}{Original Code}
LIBNAME mylib "\textcolor{OrangeRed}{X:/spileggi/Data Sets/}" ;
PROC IMPORT OUT = mylib.babies
   DATAFILE = "\textcolor{OrangeRed}{X:/spileggi/Data Sets/}babies.csv"
   DBMS = CSV REPLACE;
RUN;
\end{craw}
\emp
\vskip10pt
\bmp{0.80\textwidth}
\footnotesize
\begin{craw}{.0}{New Code}
\%LET \textcolor{OrangeRed}{mypath} = X:/spileggi/Data Sets/ ;
LIBNAME mylib "\textcolor{OrangeRed}{&mypath}" ;
PROC IMPORT OUT = mylib.babies
   DATAFILE = "\textcolor{OrangeRed}{&mypath.}babies.csv"
   DBMS = CSV REPLACE;
RUN;
\end{craw}
\emp
\bmp{0.02\textwidth} \hspace{1in} \emp
\bmp{0.25\textwidth}
A period allows you to concatenate a macro variable with other text.
\emp
\end{frame}

%===========================================================================================================================
\section[Macro modules]{Macro modules}
%===========================================================================================================================
\subsection{}
\begin{frame}
\tableofcontents[currentsection, hideallsubsections]
\end{frame}

\begin{frame}[fragile]
\frametitle{Macro Modules: Multiple Lines Insertion}
If you ever find yourself writing the same code over and over you should consider using a macro module.
\vskip10pt
\bmp{0.45\textwidth}
\begin{craw}{.0}{Macro Definition}
\%MACRO macro\_name ;

    \emph{...code...}

\%MEND ;
\end{craw}
\emp
\bmp{0.05\textwidth}\hspace{0.1in}\emp
\bmp{0.45\textwidth}
\begin{craw}{.0}{Macro Execution}


\%macro_name ;


\end{craw}
\emp
\end{frame}




\begin{frame}[fragile]
\frametitle{Macro module, no parameters}
\hspace*{-0.3in}
\bmp{0.60\textwidth}
\begin{craw}{.0}{Macro Definition}
\%MACRO myprint ;
TITLE "DATA = &dsn, OBS = &num" ;
PROC PRINT DATA = &dsn (OBS=&num);
RUN ;
TITLE ;
\%MEND ;
\end{craw}
\emp
\bmp{0.02\textwidth}\hspace{0.1in}\emp
\bmp{0.50\textwidth}
\begin{craw}{.0}{Macro Execution}

\%LET num = 10 ;
\%LET dsn = sashelp.baseball ;

\%myprint ;	

\end{craw}
\emp
\end{frame}



\begin{frame}[fragile]
\frametitle{Macro module, positional parameters }
\hspace*{-0.3in}
\bmp{0.58\textwidth}
\begin{craw}{.0}{Macro Definition}
\%MACRO myprint(dsn, num) ;
TITLE "DATA = &dsn, OBS = &num" ;
PROC PRINT DATA = &dsn (OBS=&num);
RUN ;
TITLE ;
\%MEND ;
\end{craw}
\emp
\bmp{0.02\textwidth}\hspace{0.1in}\emp
\bmp{0.52\textwidth}
\begin{craw}{.0}{Macro Executions}

\%myprint(sashelp.baseball,5) ;	

\%myprint(sashelp.class,3) ;	


\end{craw}
\emp
\begin{itemize}
\item[]
\item no equal sign in MACRO definition
\item the parameter values match the order in which they are listed in the macro definition
\item the order of the parameter values matters
\end{itemize}
\end{frame}



\begin{frame}[fragile]
\frametitle{Macro module, keyword parameters }
\hspace*{-0.35in}
\bmp{0.80\textwidth}
\begin{craw}{.0}{Macro Definition}
\%MACRO myprint(dsn = sashelp.baseball, num = 5) ;
TITLE "DATA = &dsn, OBS = &num" ;
PROC PRINT DATA = &dsn (OBS=&num);
RUN ;
TITLE ;
\%MEND ;
\end{craw}
\vskip2pt
\begin{craw}{.0}{Macro Executions}
\%myprint ;
\%myprint();
\%myprint(dsn = sashelp.baseball, num = 5) ;	
\%myprint(num = 5, dsn = sashelp.baseball) ;	
\%myprint(num = 3) ;	
\%myprint(dsn = sashelp.class, num = 3) ;
\end{craw}
\emp
%\bmp{0.01\textwidth}\hspace{0.1in}\emp
\bmp{0.34\textwidth}
\begin{itemize}
\item[]
\item uses an equal sign in MACRO definition
\item sets default values for parameters
\item can replace all or some subset of default values
\item the order of the parameter values matter does not matter
\item[]
\item[]
\item[]
\end{itemize}
\emp
\end{frame}




\begin{frame}
\ft{Developing Macro Applications}
Follow these steps to create and de-bug your SAS macros:
\begin{enumerate}
    \item Write and debug a SAS program without macro coding.
    \item Generalize the program by replacing hardcoded values with macro variable references.
    \item Create a macro definition with macro parameters.
    \item  Add macro-level programming for conditional and iterative processing.
\end{enumerate}
\end{frame}

\begin{frame}
\ft{Tips and warnings}
Tips:
\bi
\item ALWAYS include the macro debugging options in your SAS program when writing macros
\item[] \fbox{\ttt{OPTIONS MPRINT MLOGIC SYMBOLGEN;}}
\item With these options, you should be able to see what values a macro parameter resolves to.  Another way is with \ttt{\%PUT}, which prints text to the \ttt{LOG}.
\item[] \fbox{\ttt{\%PUT dsn = \&dsn , num = \&num;}}
\item[]
\ei

Warnings:
\bi
    \item use the \fbox{\ttt{\textbackslash* ... *\textbackslash}} commenting style when coding macros
    \item use double quotations (instead of single quotations) when calling macro variable names
\ei
\end{frame}

\begin{frame}
\begin{center}
\fbox{\ttt{\%LET month = January;}}
\end{center}
\begin{clicker}{Which of the following produces the title \\ \emph{The month is January}?}
    \begin{enumerate}
        \item \ttt{title "The month is \&month";}
        \item \ttt{title `The month is \&month';}
        \item \ttt{title "The month is \%month";}
        \item  \ttt{title `The month is \%month';}
    \end{enumerate}
\end{clicker}
\end{frame}

\begin{frame}
\ft{Macro Conditional Logic}
\bi
\item We can use conditional logic \textbf{outside of data steps} within macros using \ttt{\%IF, \%THEN, \%DO -- \%END, \%ELSE}
\item These statements work like their counterparts \ttt{IF, THEN, DO -- END, ELSE}
\item These conditional logic statements
	\bi
	\item \textbf{can only be used within a macro module }
	\item are `seen' only during the initial macro resolution scanning process
	\item are NOT included into the SAS code itself
	\ei
\ei
\end{frame}

\begin{frame}[fragile]
\ft{\%DO loop}
\bi
\item The \verb|%DO ... %TO| statement allows you to perform loops inside a macro. \\
\item \ttt{\%DO} loops in macros have the same kind of structure as standard \ttt{DO} loops in regular SAS code.\\
\item As with the conditional logic statements in macros \textbf{[}i.e. \verb|%IF ... %THEN|\textbf{]}, these statements \textbf{must} be embedded within a macro module -- they \textbf{cannot}  be placed outside of a macro module and in ``open SAS code''.\\
\ei
\end{frame}

\begin{frame}
\ft{Using CALL SYMPUTX}
Using \ttt{CALL SYMPUTX} allows you to take a value from the data step and assign it to a macro variable.
\begin{center}
 \fbox{\ttt{CALL SYMPUTX("macrovariablename" ,value)}}
\end{center}
	\bi
	\item \ttt{macrovariablename} \ttb{must} be surrounded by quotes
	\item \ttt{value} can be
    \bi
    \item a string in quotes (character or numeric)
    \item the name of a variable that SAS will use to assign a value (in this case, do NOT use quotes)
	\ei
%\item \ttt{CALL SYMPUTX}, introduced in Version 9, is the improved version of \ttt{CALL SYMPUT}
%	\bi
%	\item \ttt{SYMPUTX} automatically converts numeric variables to character
%variables before assigning it to a macro variable. No such conversion occurred with \ttt{SYMPUT} and that usually led to a warning message in the log.
%	\item \ttt{SYMPUTX} strips leading and trailing blanks. \ttt{SYMPUT} often included unnecessary blanks and so a function such as \ttt{TRIM} was needed ... very clumsy coding.
%	\ei
\ei
\end{frame}

%
%\begin{frame}
%\begin{clicker}{Which statement correctly creates an index variable named \ttt{i}?}
%\begin{enumerate}
%\item \ttt{\%do \&i=1 \%to 10;}
%\item \ttt{\%do \&i=1 to 10;}
%\item \ttt{\%do i=1 \%to 10;}
%\item \ttt{\%do i=1  to 10;}
%\end{enumerate}
%\end{clicker}
%\end{frame}

\end{document} 