

\input{"C:/Users/spileggi/Google Drive/STAT 330/Lectures/SlideStyle.tex"}



\title[Lecture 18]{Inputting Raw Data}
\author[Pileggi]{Shannon Pileggi}

\institute[STAT 330]{STAT 330}

\date{}


\begin{document}

\begin{frame}
\titlepage
\end{frame}

\begin{frame}
\frametitle{OUTLINE\qquad\qquad\qquad} \tableofcontents[hideallsubsections]
\end{frame}

%===========================================================================================================================
\section[List and Column Input]{List and Column Input}
%===========================================================================================================================
\subsection{}
\begin{frame}
\ft{Overview}
So far we have learned three methods for working with data in SAS:
\begin{enumerate}
\item use a \ttt{LIBNAME} statement to access a SAS library that contains SAS data
\item use \ttt{DATALINES} in a DATA step to enter data (this is an example of \emph{list input} with instream data)
\item use \ttt{PROC IMPORT} for structured data files like CSV or EXCEL
\item[]
\end{enumerate}
Today - importing ``raw''/less structured data (think \ttt{.txt} or \ttt{.dat} extension) with \ttt{INFILE}.
\vskip5pt
\end{frame}

\begin{frame}
\ft{The Pointer}
\bi
\item SAS has a virtual \tte{pointer} which keeps track of the `reading' location in the data file
\item Pointer location depends upon method of data input
\begin{enumerate}
\item List - pointer moves to next non-empty column to begin reading
\item Column - pointer moves to the explicitly designated column location
\item Formatted - pointer moves to column depending upon specified informat length
\end{enumerate}
\ei
\end{frame}

\begin{frame}[fragile]
\ft{INFILE statement}
\bmp{1.0\textwidth}
\footnotesize
\begin{code}{.0}
DATA \emph{mydata} ;
   INFILE "C:/ComputerLocation/\emph{datasetname.ext}" ;
   INPUT \emph{var1} \emph{var2} \emph{var3};
RUN;
\end{code}
\emp
\vskip5pt
\bi
\item the \ttt{INFILE} statement specifies the computer location of the data file
\bi
\item be sure to include the data set name at the end of the path
\item be sure to include the data set extension at the end of the path
\item the path goes in quotes
\ei
\item the \ttt{INPUT} statement names the variables
\ei
\end{frame}

%\begin{frame}
%\ft{An aside...}
%You can let a computer location be a macro variable.  \\
%\fbox{\ttt{\footnotesize{\%let path=C:/Users/spileggi/Google Drive/STAT330/Data/;}}}\\
%\vskip10pt
%Now this SAS statement\\
%\fbox{\ttt{\footnotesize{"\&path.Data\_cheese.dat"}}}\\
%\vskip10pt
%resolves to\\
%\fbox{\ttt{\footnotesize{"C:/Users/spileggi/Google Drive/STAT330/Data/Data\_cheese.dat"}}}\\
%\vskip10pt
%(To combine a text string with a macro variable, separate the two with a period.)
%\vskip10pt
%\oyo Why would this be useful?
%\end{frame}

\begin{frame}[fragile]
\ft{About the cheese data}
In a study of cheddar cheese from the LaTrobe Valley of Victoria, Australia, samples of cheese were analyzed for their chemical composition and were subjected to taste tests.\\
\vskip10pt
\hspace*{-0.3in}
\bmp{0.5\textwidth}
\resizebox{1.0\textwidth}{!}{
\begin{tabular}{r|l}
\ttt{case} & sample number \\
\ttt{taste}  & subjective averaged taste \\ & test score \\
\ttt{acetic} & natural log of concentration \\ &  of acetic acid \\
\ttt{h2S}    & natural log of concentration \\ & of hydrogen sulfide \\
\ttt{lactic} & concentration of lactic acid \\
\end{tabular}}
\emp
\blankcolumn
\bmp{0.5\textwidth}
\begin{craw}{.0}{Data\_cheese.dat}
1  12.3 4.543   3.135   0.86
2  20.9 5.159   5.043   1.53
3  39   5.366   5.438   1.57
4  47.9 5.759   7.496   1.81
5  5.6  4.663   3.807   0.99
\end{craw}
\emp
%\vskip10pt
%\oyo Open the data in a text editor like notepad.  Do the data look messy or neat?
\end{frame}

\begin{frame}[fragile]
\ft{Importing the cheese data}
\bmp{1.0\textwidth}
\footnotesize
\begin{code}{.0}
DATA cheese;
   INFILE "\&path.Data\_cheese.dat";
   INPUT case taste acetic h2s lactic;
RUN;
\end{code}
\emp
\vskip10pt
\begin{clicker}{Which method determined the pointer location?}
\begin{enumerate}
\item list
\item column
\item formatted
\item none of these
\end{enumerate}
\end{clicker}
\end{frame}

\begin{frame}
\ft{List input limitations}
We can't use list input when we
\bi
\item do not have a delimiter (ie, no space or comma or something) between values
\item do not have periods for missing values
\item have non-standard data (ie, dates)
\item have data with embedded spaces
\item have character variables with width $>8$  characters
\ei
\end{frame}

\begin{frame}[fragile]
\ft{Column input}
\begin{itemize}
\item limitations: data must be lined up in the same columns
\item advantages: can work with embedded spaces, character variables $>8$, missing data indicated by spaces
\end{itemize}
\vskip5pt
\bmp{0.65\textwidth}
\footnotesize
\begin{craw}{.0}{Data\_cheese.dat}
1       12.3    4.543   3.135   0.86
2       20.9    5.159   5.043   1.53
3       39      5.366   5.438   1.57
4       47.9    5.759   7.496   1.81
5       5.6     4.663   3.807   0.99
\end{craw}
\emp
\bmp{0.05\textwidth} \hspace{0.05in}\emp
\bmp{0.30\textwidth}
\begin{clicker}{Can we use column input for the cheese data?}
\begin{enumerate}
\item Yes
\item No
\end{enumerate}
\end{clicker}
\emp
\end{frame}

\begin{frame}[fragile]
\ft{Column input}
\bmp{0.65\textwidth}
\footnotesize
\begin{craw}{.0}{Data\_cheese.dat}
0        1         2         3
123456789012345678901234567890123456
------------------------------------
1       12.3    4.543   3.135   0.86
2       20.9    5.159   5.043   1.53
3       39      5.366   5.438   1.57
4       47.9    5.759   7.496   1.81
5       5.6     4.663   3.807   0.99
\end{craw}
\emp
\bmp{0.05\textwidth} \hspace{0.05in} \emp
\bmp{0.35\textwidth}
\begin{clicker}{In which columns is the second variable (\ttt{taste}) located?}
\begin{enumerate}
\item 2 through 9
\item 8 through 13
\item 9 through 12
\item 9 through 16
\item 6 though 16
\end{enumerate}
\end{clicker}
\emp
\end{frame}

\begin{frame}[fragile]
\ft{Column input example}
\bmp{1.05\textwidth}
\footnotesize
\begin{code}{.0}
DATA cheese2;
   INFILE "\&path.Data\_cheese.dat";
   INPUT case 1-2 taste 9-12 acetic 17-21 h2s 25-29 lactic 33-36;
RUN;
\end{code}
\emp
\bi
\item[]
\item \emph{after} each variable name, specify the numeric range of column position
\ei
\end{frame}

\begin{frame}[fragile]
\ft{Discussion}
\bmp{1.05\textwidth}
\footnotesize
\begin{code}{.0}
*Example 3 - change input order;
DATA cheese3;
   INFILE "\&path.Data\_cheese.dat";
   INPUT \textcolor{OrangeRed}{lactic 33-36} case 1-2 taste 9-12 acetic 17-21 h2s 25-29;
RUN;

*Example 4 - mix input methods;
DATA cheese4;
   INFILE "\&path.Data\_cheese.dat";
   INPUT \textcolor{OrangeRed}{case taste}  acetic 17-21 h2s 25-29 lactic 33-36;
RUN;
\end{code}
\emp
\vskip10pt
\oyo Will these examples work or will there be an error?
\end{frame}

%===========================================================================================================================
\section[Formatted Input]{Formatted Input}
%===========================================================================================================================
\subsection{}
\begin{frame}
\tableofcontents[currentsection, hideallsubsections]
\end{frame}

\begin{frame}
\ft{Review}
Informats are used to read non-standard data.
\vskip10pt
\bt{ll}
Character: &  \ttt{\$name\myuscore of\myuscore informat}$w.$\\
Numeric:  &  \ttt{name\myuscore of\myuscore informat}$w.d$\\
Date:  & \ttt{name\myuscore of\myuscore informat}$w.$
\et
\vskip10pt
\bi
\item  $w$ is the width of the \emph{entire} field, including special characters
\item  $d$ is the number of decimals
\item $.$ period indicates that we are establishing an informat (rather than a variable name)
\ei
\vskip10pt
\emph{Note:} the default width for character variables is 8.
\end{frame}

\begin{frame}[fragile]
\ft{Discussion}
The \ttt{DOLLAR}$w.d$ informat removes embedded characters for numeric data.
\vskip10pt
\bmp{0.60\textwidth}
\footnotesize
\begin{code}{.0}
DATA test;
INPUT name \$11. money \textcolor{OrangeRed}{?} ;
DATALINES;
Constantine \$15,000.35
Billy       \$8,000.05
Sue         \$3,000.63
Megan       \$400.45
;
RUN;
\end{code}
\emp
\bmp{0.05\textwidth}
\hspace{0.05in}
\emp
\bmp{0.40\textwidth}
\begin{clicker}{Which is the correct informat for \ttt{money}?}
\begin{enumerate}
    \item \ttt{DOLLAR}5.2
    \item \ttt{DOLLAR}7.2
    \item \ttt{DOLLAR}10.2
    \item \ttt{DOLLAR}12.0
\end{enumerate}
\end{clicker}
\emp
\end{frame}

\begin{frame}[fragile]
\ft{The pointer with formatted input}
\bmp{0.60\textwidth}
\footnotesize
\begin{code}{.0}
DATA test;
INPUT name \$11. money DOLLAR10.2 ;
DATALINES;
Constantine \$15,000.35
Billy       \$8,000.05
Sue         \$3,000.63
Megan       \$400.45
;
RUN;
\end{code}
\emp
\bmp{0.05\textwidth}
\hspace{0.05in}
\emp
\bmp{0.40\textwidth}
\bi
\item SAS looks for \ttt{name} in columns 1 through 11.
\item The pointer moves to column 12.
\item SAS looks for \ttt{money} in columns 13 through 22.
\ei
\emp
\end{frame}

\begin{frame}[fragile]
\ft{Discussion}
\bmp{0.60\textwidth}
\footnotesize
\begin{code}{.0}
DATA test2;
INPUT name \$11. money DOLLAR10.2 ;
DATALINES;
Constantine \$15,000.35
Billy \$8,000.05
Sue \$3,000.63
Megan \$400.45
;
RUN;
\end{code}
\emp
\bmp{0.05\textwidth}
\hspace{0.05in}
\emp
\bmp{0.40\textwidth}
\begin{clicker}{If my data look like this, will it still import correctly?}
\begin{enumerate}
\item Yes
\item No
\end{enumerate}
\end{clicker}
\emp
\end{frame}

\begin{frame}
\ft{The colon modifier}
\bi
\item A colon modifies an informat
\item examples:  \fbox{\ttt{:COMMA6.}}    \fbox{\ttt{:MMDDYY10.}}    \fbox{\ttt{:\$10.}}
\item A colon allows you to use an informat for reading data in an otherwise list input process.  Why?
\bi
\item If you assign an informat like \fbox{\ttt{\$10.}}, SAS will read 10 columns every time, and may include unwanted characters
\ei
\item Applying the colon modifier tells SAS to read a value \emph{until} it encounters a space (so it doesn't use a set column position)
\ei
%\vskip10pt
%\oyo How can we correctly read in the \ttt{test} data from the discussion slide?
\end{frame}

\begin{frame}
\ft{The ampersand modifier}
\bi
\item An ampersand modifies an informat
\item examples:  \fbox{\ttt{\&COMMA6.}}    \fbox{\ttt{\&MMDDYY10.}}    \fbox{\ttt{\&\$10.}}
\item Continues to read a character value, even if it contains blanks
\item Two or more blanks indicates the data value is complete
\ei
\vskip10pt
\oyo How can we correctly read in the \ttt{test} data from the previous slide?
%INPUT name :$11. money DOLLAR10.2 ;
\end{frame}


\begin{frame}
\ft{Moving the pointer}
Formatted data can cause issues with the pointer because it forces the pointer to look in specific columns.  You can manually move the pointer by specifying pointer location \underline{before} the variable name.
\bi
\item[]
\item \ttt{+n} -- Move pointer ahead $n$ columns
\item \ttt{@n} -- Move pointer directly to column $n$
\ei
\end{frame}

\begin{frame}[fragile]
\ft{House data}
\bmp{0.35\textwidth}
\resizebox{1.0\textwidth}{!}{
\begin{tabular}{r|l}
\ttt{price} & selling price \\
\ttt{size} & square feet of home \\
\ttt{age} & age of home \\
\ttt{numfeat} & number of features \\
\ttt{NEloc} & NorthEast Location \\
\ttt{CustLab} & Customer Label \\
\ttt{CorLoc} & Corner Location \\
\ttt{tax} & Annual taxes \\
\end{tabular}}
\emp
\blankcolumn
\bmp{0.65\textwidth}
\oyo What features do you notice in the data?
\emp
\vskip5pt
\hspace*{-0.3in}
\bmp{1.15\textwidth}
\begin{scriptcraw}{.0}{Data\_homeprice.dat}
205,000    2650 sqft   13      7       1       1       0       1639
208,000    2600 sqft   .      4       1       1       0       1088
215,000    2664 sqft   6      5       1       1       0       1193
215,000    2921 sqft   3      6       1       1       0       1635
199,900    2580 sqft   4      4       1       1       0       1732
190,000    2580 sqft   4      4       1       0       0       1534
180,000    2774 sqft   2      4       1       0       0       1765
156,000    1920 sqft   1      5       1       1       0       1161
145,000    2150 sqft   .      4       1       0       0          .
144,900    1710 sqft   1      3       1       1       0       1010
137,500    1837 sqft   4      5       1       0       0       1191
127,000    1880 sqft   8      6       1       0       0        930
125,000    2150 sqft   15      3       1       0       0        984
\end{scriptcraw}
\emp
\end{frame}


\begin{frame}[fragile]
\ft{Discussion}
\bmp{1.00\textwidth}
\footnotesize
\begin{code}{.0}
DATA homes;
INFILE "\&path.Data_homeprice.dat";
INPUT price COMMA7. size 12-15 @23 age
      numfeat NEloc CustLab CorLoc tax;
RUN;
\end{code}
\emp
\vskip10pt
\begin{clicker}{Which method(s) did I use to import the data?}
\begin{enumerate}
\bmp{0.4\textwidth}
\item list input
\item column input
\item formatted input
\item[]
\emp
\bmp{0.5\textwidth}
\item list $+$ column
\item list $+$ formatted
\item column $+$ formatted
\item list $+$ column $+$ formatted
\emp
\end{enumerate}
\end{clicker}
\end{frame}

\begin{frame}
\ft{Wrap up}
\bi
\item Importing data correctly may take some trial and error and there can be multiple correct methods.
\item \textcolor{OrangeRed}{TIP:} the pointer moves from left to right.  Variable 10 will likely not appear correctly if Variable 1 does not appear correctly.  Get variables to appear correctly, one at time, according to their input order.
\item \textcolor{OrangeRed}{TIP:} Always check \ttt{PROC CONTENTS} to make sure that variables have the correct type (numeric vs character).
\item \textcolor{OrangeRed}{TIP:} If data file is open in a separate application, you may need to close it before you import it.
\item Input features must go before/after variable names
\item[]
\item[]
\begin{tabular}{lcc}
Feature & Before  & After  \\
\hline
Column position (e.g., \ttt{3-4}) & \rx & \gc \\
Informat (e.g., \ttt{COMMA7.2}) & \rx & \gc \\
\ttt{+n}/\ttt{@n} & \gc & \rx \\
\end{tabular}
\ei
\end{frame}

%\begin{frame}
%\ft{Informat modifiers}
%\bi
%\item You already learned the colon modifier, which tells SAS to read a value \emph{until} it encounters a space
%\item[] examples:  \fbox{\ttt{:COMMA6.}}    \fbox{\ttt{:MMDDYY10.}}    \fbox{\ttt{:\$10.}}
%\item Another is the ampersand modifier, which can be used to read character values that contain embedded blanks
%\item[] examples:  \fbox{\ttt{\&COMMA6.}}    \fbox{\ttt{\&MMDDYY10.}}    \fbox{\ttt{\&\$10.}}
%\item If you assign an informat like \fbox{\ttt{\$10.}}, SAS will read 10 columns every time, and may include unwanted characters
%\item Applying a modifier is useful when data do not come in fixed fields or fixed lengths
%\ei
%\end{frame}

%===========================================================================================================================
\section[More on moving the pointer]{More on moving the pointer}
%===========================================================================================================================
\subsection{}
\begin{frame}
\tableofcontents[currentsection, hideallsubsections]
\end{frame}


\begin{frame}[fragile]
\ft{\ttt{@"XXXX"} column pointer}
\begin{craw}{.1}{Data\_Dogs.dat}
  Name- Kia Breed: Shepherd Vet Bills: 325.25
     Name- Sam Breed: Beagle   Vet Bills: 478.78
   Name- Sydney    Breed: Boxer   Vet Bills: 733.54
 Name- Bugsy   Breed: Pug  Vet Bills: 518.09
\end{craw}
\bmp{0.45\textwidth}
\bi
\item[]
\item When data have a consistent prefix use the \ttt{@"XXXX"} column pointer
\item \ttt{XXXX} represents the prefix
\ei
\emp
\bmp{0.05\textwidth} \hspace{0.05in} \emp
\bmp{0.55\textwidth}
\begin{clicker}{What are the \emph{exact} prefixes for...}
\begin{enumerate}
\item Name of the dog
\item Breed of the dog
\item Amount spent at the vet
\end{enumerate}
\end{clicker}
\emp
\end{frame}


\begin{frame}[fragile]
\ft{Importing the dog data}
\bmp{1.00\textwidth}
\footnotesize
\begin{code}{.0}
DATA dogs;
INFILE "\&path.Data_dogs.dat";
INPUT @"Name- " name \$
      @"Breed: " breed \$
      @"Bills: " spending COMMA6.2;
RUN;
\end{code}
\emp
\end{frame}




\begin{frame}[fragile]
\ft{Address data}
\bmp{0.70\textwidth}
\footnotesize
\begin{craw}{.0}{mailing.dat}
Brenda Smith  email: Bsmith@charter.net
123 Grand Ave.
Arroyo Grande  CA 93420
David White  email: david6060@gmail.com
456 Traffic Wy.
Arroyo Grande  CA 93420
Alexandra Jones  email: AJJ43@yahoo.com
789 Foothill Blvd.
San Luis Obispo  CA 93405
\end{craw}
\emp
\bmp{0.05\textwidth} \hspace{0.05in} \emp
\bmp{0.30\textwidth}
\oyo What features does this data have that we will need to address when importing?
\emp
\end{frame}

\begin{frame}
\ft{Line pointers}
\bi
\item Raw data sets typically consist of one observation per line
%\item SAS goes to next line if it runs out of data before it has read all variables from input statement
\item Line pointers tells SAS to to skip to a new line
\bi
\item \fbox{\ttt{/}} skip to next line
\item \fbox{\ttt{\#n}} skip to line \ttt{n}
\ei
\item This is used to read multiple lines of data into a single observation
\ei
\end{frame}

\begin{frame}[fragile]
\ft{Reading the mail data}
\bmp{1.0\textwidth}
\footnotesize
\begin{code}{.0}
DATA mailing;
INFILE "&path.mailing.dat";
INPUT fname :\$10. lname \$ @"email: " email :\$20.
      / street \&\$30.
      / city \&\$30. state \$2. zip ;

/*this also works:
INPUT fname :\$10. lname \$ @"email: " email :\$20.
      #2 street \&\$30.
      #3  city \&\$30. state \$2. zip ;
*/
RUN;

\end{code}
\emp
\end{frame}


\begin{frame}
\ft{@ vs @@}
\bmp{0.5\textwidth}
Double trailing at \fbox{\ttt{@@}}
\bi
\item SAS assumes that a single line of data corresponds to a single observation
\item If a single line of a data corresponds to multiple observations, need to use \fbox{\ttt{@@}}
\item Tells SAS to keep reading data into new observations until it runs out
\ei
\emp
\bmp{0.05\textwidth} \hspace{0.05in}\emp
\bmp{0.5\textwidth}
Trailing at \fbox{\ttt{@}}
\bi
\item Use when interested in specific records from raw data
\item Tells SAS to wait for more information
\item Syntax is typically
\bi
    \item input statement with @
    \item if-then statement to select obs
    \item new input statement
\ei
\ei
\emp
\end{frame}


\begin{frame}[fragile]
\ft{@@ example}
\bmp{1.0\textwidth}
\begin{craw}{.0}{baggagefees.dat}
1  Delta  863,608  2  American  593,465
3  US Airways  506,339  4  Continental  353,416
5  United  276,817  6  AirTran  164,670
7  Alaska  157,013  8  Spirit  133,970
9  JetBlue  64,078  10  Hawaiian  56,590
11  Frontier  54,862  12  Allegiant  53,562
13  Virgin America  33,482  14  Southwest  32,035
15  Sun Country  13,398  16  Mesa  1,683
17  USA 3000  1,650
\end{craw}
\vskip3pt
\begin{code}{.0}
DATA baggage ;
   INFILE "\&path.baggagefees.dat";
   INPUT rank airline \&\$20. revenue :COMMA. @@;
RUN;
\end{code}
\emp
\end{frame}


\begin{frame}[fragile]
\ft{@ example}
\bmp{0.7\textwidth}
\begin{craw}{.0}{potassium.dat}
Mollusk, clams        534  85  3 oz.
Cod                   439  85  3 oz.
Halibut               490  85  3 oz.
Salmon                319  85  3 oz.
Trout                 375  85  3 oz.
Tuna                  484  85  3 oz.
Apricots, dried       814  70  10 med.
\end{craw}
\emp
\blankcolumn
\bmp{0.3\textwidth}
Objective: retain observations with potassium (K) greater than 500
\emp
\vskip3pt
\bmp{1.0\textwidth}
\begin{code}{.0}
DATA potassium ;
   INFILE "\&path.potassium.dat" ;
   INPUT @21 K COMMA5. @ ;
   IF K < 500 THEN DELETE ;
   INPUT food \$ 1-20 @28 weight measure \&\$10.;
RUN;
\end{code}
\emp
\end{frame}

\begin{frame}
\ft{Discussion}
\begin{clicker}{Which symbol best represents the \emph{opposite} operation\\ performed by \ttt{@@}?}
\begin{enumerate}
\item \ttt{@}
\item \ttt{@"XXX"}
\item \ttt{@n, +n}
\item \ttt{/, \#n}
\item \ttt{\&}
\item \ttt{:}
\end{enumerate}
\end{clicker}
\end{frame}

%===========================================================================================================================
\section[INFILE]{INFILE}
%===========================================================================================================================
\subsection{}
\begin{frame}
\tableofcontents[currentsection, hideallsubsections]
\end{frame}

\begin{frame}
\ft{INFILE options}
\bi
\item \fbox{\ttt{FIRSTOBS=}} tells SAS at which line to \emph{start} reading
\item[] useful to skip variable names
\item \fbox{\ttt{OBS=}} tells SAS at which line to \emph{stop} reading
\item[] useful to read in part of data file
\item \fbox{\ttt{FLOWOVER}} default value; SAS jumps to next line if current line does not have enough values to read.  After a jump, SAS reads the next line regardless of whether it has enough values.
\item \fbox{\ttt{MISSOVER}} Set a variable to missing value if missing or if length is too short
\item \fbox{\ttt{TRUNCOVER}} Allows SAS to handle data values of varying lengths appropriately with column or formatted input
\ei
\end{frame}

\begin{frame}[fragile]
\ft{Try it}
\bmp{0.5\textwidth}
\begin{craw}{.0}{names.txt}
FirstName LastName
Allison Allen
Billy Bryson
Carmen Cottle
David Decker
Enrique Edwards
Faith Firth
\end{craw}
\emp
\bmp{0.05\textwidth} \hspace{0.05in} \emp
\bmp{0.45\textwidth}
\begin{clicker}{What values should I use below to read in the A through D names?}
\begin{enumerate}
\item 1, 5
\item 2, 5
\item 1, 4
\item 2, 4
\item 4, 2
\end{enumerate}
\end{clicker}
\emp

\bmp{1.0\textwidth}
\begin{code}{.0}
DATA names ;
  INFILE "\&path.names.txt" FIRSTOBS=\fbox{\textcolor{White}{2}} OBS=\fbox{\textcolor{White}{5}};
  INPUT fname \$ lname \$;
RUN;
\end{code}
\emp



\end{frame}


\begin{frame}[fragile]
\ft{Try it}
\bmp{0.65\textwidth}
\footnotesize
\begin{code}{.0}
\%MACRO checkoptions(option);
DATA test;
   INFILE "\&path.numbers.txt" \&option;
   INPUT numbers \textcolor{OrangeRed}{6.};
RUN;
TITLE "\&option";
PROC PRINT; RUN;
\%MEND;

\%checkoptions(flowover);
\%checkoptions(missover);
\%checkoptions(truncover);
\end{code}
\emp
\bmp{0.05\textwidth} \hspace{0.05in} \emp
\bmp{0.3\textwidth}
\begin{craw}{.0}{numbers.txt}
666666
1
22
333
4444
55555
\end{craw}
\vskip3pt
\oyo Try changing the \fbox{\ttt{\textcolor{OrangeRed}{6.}}} numeric format to
\begin{enumerate}
\item \fbox{\ttt{\textcolor{White}{.6}}} (nothing)
\item \fbox{\ttt{:6.}}
\end{enumerate}
What differences do you observe?
\emp
\end{frame}

\begin{frame}
\frametitle{Delimited files}
\bi
\item A delimited file contains a specific character that separates data values
\item with list input, SAS assumes the delimiter is a space
\item \fbox{\ttt{DLM=}} option in the \ttt{infile} statement specifies the delimiter
\bi
\item \fbox{\ttt{DLM=`,'}} for commas
\item \fbox{\ttt{DLM=`09'x}} for tabs
\ei
\item by default, SAS assumes that $\ge 2$ delimiters in a row is a single delimiter; to override this, the \fbox{\ttt{DSD}} option:
\bi
\item treats two delimeters in a row as a missing value
\item ignores delimiters enclosed in quotes
\item does not read quotes as part of the data value
\ei
\ei
\end{frame}

\begin{frame}[fragile]
\ft{Example}
\bmp{1.0\textwidth}
\begin{craw}{.0}{beer.csv}
Consumption per capita [1],,,
Country,Consumption (liters),2009–2010 (change 633-ml bottles),Total national consumption (106 L)
Vietnam,19,,
Venezuela,83,-4.7,2259
Uzbekistan,11,,
United States,78,-2.5,24138
United Kingdom,74,-3.4,4587
\end{craw}
\vskip3pt
\begin{code}{.0}
DATA beer;
   INFILE "\&path.beer.csv" DSD FIRSTOBS=3 DLM=",";
   INPUT country :\$30. consumption change total;
RUN;
\end{code}
\emp
%\vskip10pt
%\emph{Note: you must close the csv file in excel before you import the data in SAS.}
\end{frame}


%\begin{frame}[fragile]
%\ft{Example}
%\oyo Examine the \ttt{beer.csv} in
%\begin{enumerate}
%\item excel
%\item notepad
%\end{enumerate}
%What features does the data have that we will need to address?
%\vskip10pt
%\bmp{1.0\textwidth}
%\footnotesize
%\begin{code}{.0}
%DATA beer;
%   INFILE "\&path.beer.csv" DSD FIRSTOBS=3 DLM=",";
%   INPUT country :\$30. consumption change total;
%RUN;
%\end{code}
%\emp
%\vskip10pt
%\emph{Note: you must close the csv file in excel before you import the data in SAS.}
%\end{frame}




\end{document} 