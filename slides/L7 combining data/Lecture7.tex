

\input{"C:/Users/spileggi/Google Drive/STAT 330/Lectures/SlideStyle.tex"}



\title[Lecture 7]{Combining Data}
\author[Pileggi]{Shannon Pileggi}

\institute[STAT 330]{STAT 330}

\date{}


\begin{document}

\begin{frame}
\titlepage
\end{frame}

\begin{frame}
\frametitle{OUTLINE\qquad\qquad\qquad} \tableofcontents[hideallsubsections]
\end{frame}

%\begin{frame}
%\ft{Overview}
%\begin{tabular}{p{3cm} p{3cm} p6{cm}}
%Action & Description & SAS key statements \\
%Concatenate & Stack data steps on top of each other
%Interleave &
%One to One Merge &
%One to Many Merge &
%\end{tabular}
%\end{frame}

%===========================================================================================================================
\section[Stacking]{Stacking}
%===========================================================================================================================
\subsection{}
\begin{frame}[fragile]
\ft{Concatenate}
\bi
\item[Goal:] combine multiple data sets that have the same variables
\item[How:] use the \ttt{SET} statement to concatenate (or stack) the data sets
\ei
\bmp{.50\textwidth}
\footnotesize
\begin{craw}{.1}{Data Set MEN}
ID   NAME     GRADE
1    Andrew     B+
3    Jimmy      B-
5    Ulric      A-
\end{craw}
\vspace{.0in}
\begin{craw}{.1}{Data Set WOMEN}
ID   NAME     GRADE
2    Soma       B
4    Karen      A
6    Beth       B+
\end{craw}
\vspace{.0in}
\emp
\bmp{.50\textwidth}
\begin{code}{.}
DATA STACKED;
   \textcolor{OrangeRed}{SET} men women;
RUN;
\end{code}
\vspace{.0in}
\begin{craw}{.1}{Data Set STACKED}
ID   NAME     GRADE
1    Andrew     B+
3    Jimmy      B-
5    Ulric      A-
2    Soma       B
4    Karen      A
6    Beth       B+
\end{craw}
\emp
\vspace{.0in}
\end{frame}

\begin{frame}[fragile]
\ft{Interleave}
\bi
\item[Goal:] stack data while retaining some sort of order
\item[How:] use the \ttt{SET} statement with \ttt{BY} (data must be pre-sorted)
\ei
\bmp{.50\textwidth}
\footnotesize
\begin{craw}{.1}{Data Set MEN}
ID   NAME     GRADE
1    Andrew     B+
3    Jimmy      B-
5    Ulric      A-
\end{craw}
\vspace{.0in}
\begin{craw}{.1}{Data Set WOMEN}
ID   NAME     GRADE
2    Soma       B
4    Karen      A
6    Beth       B+
\end{craw}
\vspace{.0in}
\emp
\bmp{.50\textwidth}
\begin{code}{.}
DATA INTERLEAVE;
   \textcolor{OrangeRed}{SET} men women;
   \textcolor{OrangeRed}{BY} ID;
RUN;
\end{code}
\vspace{.0in}
\begin{craw}{.1}{Data Set INTERLEAVE}
ID   NAME     GRADE
1    Andrew     B+
2    Soma       B
3    Jimmy      B-
4    Karen      A
5    Ulric      A-
6    Beth       B+
\end{craw}
\emp
\vspace{.0in}
\end{frame}

\begin{frame}[fragile]
\ft{PROC SORT syntax}
\bmp{0.45\textwidth}
\footnotesize
\begin{code}{.0}
PROC SORT DATA = men ;
   BY ID ;
RUN ;

PROC SORT DATA = women ;
   BY ID ;
RUN ;

DATA DEMO;
SET men women ;
BY ID;
RUN;
\end{code}
%\vskip5pt
Original \texttt{men} and \texttt{women} data set \\are sorted.
\vskip40pt
\textcolor{white}{x}
\emp
\bmp{0.02\textwidth}\hspace{1in} \emp
\bmp{0.55\textwidth}
\footnotesize
\begin{code}{.0}
PROC SORT DATA = men
   \textcolor{OrangeRed}{OUT = sorted_men} ;
   BY ID ;
RUN ;
PROC SORT DATA = women
   \textcolor{OrangeRed}{OUT = sorted_women} ;
   BY ID ;
RUN ;
DATA DEMO;
SET \textcolor{OrangeRed}{sorted_men} \textcolor{OrangeRed}{sorted_women} ;
BY ID;
RUN;
\end{code}
%\vskip5pt
Original \texttt{men} and \texttt{women} data sets remain \emph{unsorted}; newly created data sets \textcolor{OrangeRed}{\texttt{sorted\_men}} and \textcolor{OrangeRed}{\texttt{sorted\_women}} are sorted.
\vskip30pt
\textcolor{white}{x}
\emp
\end{frame}

\begin{frame}[fragile]
\ft{Discussion}
\bmp{.50\textwidth}
\footnotesize
\begin{craw}{.1}{Data Set MEN}
ID   NAME     \textcolor{OrangeRed}{GRADE}
1    Andrew     B+
3    Jimmy      B-
5    Ulric      A-
\end{craw}
\vspace{.0in}
\begin{craw}{.1}{Data Set WOMEN}
ID   NAME     \textcolor{OrangeRed}{LETTER}
2    Soma       B
4    Karen      A
6    Beth       B+
\end{craw}
\vspace{.0in}
\emp
\bmp{.57\textwidth}
\begin{code}{.}
DATA demo;
  SET men women ;
RUN;
\end{code}
\vspace{.0in}
\begin{clicker}{The goal is for the \texttt{demo} data set to have \underline{\hspace{0.3in}} variables; it would have \underline{\hspace{0.3in}} variables.}
\begin{enumerate}
\item 2, 3
\item 3, 3
\item 3, 4
\item 4, 3
\item 4, 4
\end{enumerate}
\end{clicker}
\emp
\vspace{.0in}
\end{frame}


\begin{frame}[fragile]
\ft{Rename}
\bi
\item[Goal:] stack data with same variables but different names
%\item[How:] use the \ttt{rename} statement
\ei
\bmp{.50\textwidth}
\footnotesize
\begin{craw}{.1}{Data Set MEN}
ID   NAME     \textcolor{OrangeRed}{GRADE}
1    Andrew     B+
3    Jimmy      B-
5    Ulric      A-
\end{craw}
\vspace{.0in}
\begin{craw}{.1}{Data Set WOMEN}
ID   NAME     \textcolor{OrangeRed}{LETTER}
2    Soma       B
4    Karen      A
6    Beth       B+
\end{craw}
\vspace{.0in}
\emp
\bmp{.57\textwidth}
\begin{code}{.}
DATA DEMO;
SET men
  women \textcolor{OrangeRed}{(RENAME=(LETTER=GRADE))};
BY ID;
RUN;
\end{code}
\vspace{.0in}
\begin{craw}{.1}{Data Set DEMO}
ID   NAME     GRADE
1    Andrew     B+
2    Soma       B
3    Jimmy      B-
4    Karen      A
5    Ulric      A-
6    Beth       B+
\end{craw}
\emp
\vspace{.0in}
\end{frame}







%===========================================================================================================================
\section[Merging]{Merging}
%===========================================================================================================================
\subsection{}
\begin{frame}
\tableofcontents[currentsection, hideallsubsections]
\end{frame}


\begin{frame}[fragile]
\ft{One to one merge}
\bi
\item[Goal:] \small{combine multiple data sets that have some related and some different variables}
\item[How:] \small{use the \ttt{MERGE} statement \ttt{BY} identifying variables (data must be pre-sorted)}
\ei
\bmp{.50\textwidth}
\footnotesize
\begin{craw}{.1}{Data Set MEN}
ID   NAME     GRADE
1    Andrew     B+
3    Jimmy      B-
5    Ulric      A-
\end{craw}
%\vspace{.0in}
\begin{craw}{.1}{Data Set HEIGHT\_M}
ID HEIGHT
1  68
3  69
5  72
\end{craw}
\vspace{.0in}
\emp
\bmp{.50\textwidth}
\begin{code}{.}
DATA merged;
  \textcolor{OrangeRed}{MERGE} men height\_m;
  \textcolor{OrangeRed}{BY} ID;
RUN;
\end{code}
%\vspace{.0in}
\begin{craw}{.1}{Data Set MERGED}
ID NAME   GRADE HEIGHT
1  Andrew  B+    68
3  Jimmy   B-    69
5  Ulric   A-    72
\end{craw}
\emp
\vspace{.0in}
\end{frame}



\begin{frame}[fragile]
\ft{Discussion}
\bmp{.50\textwidth}
\footnotesize
\begin{craw}{.1}{Data Set MEN}
ID   NAME     \textcolor{OrangeRed}{GRADE}
1    Andrew   \textcolor{OrangeRed}{B+}
3    Jimmy    \textcolor{OrangeRed}{B-}
5    Ulric    \textcolor{OrangeRed}{A-}
\end{craw}
%\vspace{.0in}
\begin{craw}{.1}{Data Set HEIGHT\_M}
ID HEIGHT \textcolor{OrangeRed}{GRADE}
1  68     \textcolor{OrangeRed}{F}
3  69     \textcolor{OrangeRed}{F}
5  72     \textcolor{OrangeRed}{F}
\end{craw}
\vspace{.0in}
\emp
\bmp{.50\textwidth}
\begin{code}{.}
DATA merged;
   MERGE men height\_m;
   BY ID;
RUN;
\end{code}
%\vspace{.0in}
\oyo
\begin{enumerate}
\item How many variables will be in the resulting data set?
\item What will be the values of the \ttt{GRADE} variable(s)?
\item[]
\end{enumerate}
\emp
\vspace{.0in}
\end{frame}


\begin{frame}[fragile]
\ft{Merging issues}
\bi
\item Must have at least one common variable between the data sets to use for matching purposes (like ID)
\item Data sets need to be pre-sorted by the variable(s) specified in the \ttt{BY} statement
\item When merging two data sets that have a variable name in common (which is not an identifying variable) the variable from the second data set will
\textbf{overwrite} the first
\item To fix this, use data set options (like drop/keep/rename) \underline{in parentheses} beside the data set name
\ei
\end{frame}


\begin{frame}
\ft{Data set options}
\bmp{0.4\textwidth} \hspace{0.05\in} \emp
\bmp{0.6\textwidth}
\bi
\item[\fbox{\ttt{KEEP =} \emph{variable-list}}] specifies variable(s) to keep
\item[\fbox{\ttt{DROP =} \emph{variable-list}}] specifies variable(s) to drop
\item[\fbox{\ttt{RENAME =} \emph{(oldvar=newvar)}}] renames variable(s)
\item[\fbox{\ttt{FIRSTOBS =} \emph{n}}] start reading at $n$
\item[\fbox{\ttt{OBS =} \emph{n}}] stop reading at $n$
\item[\fbox{\ttt{IN =} \emph{new-var-name}}] creates temporary tracking variable
\item[\fbox{\ttt{WHERE =} \emph{condition}}] selects observations
\item[]
\ei
\emp \\
\hspace*{-0.3in}
\bi
\item[Ex1] \ttt{SET animals (KEEP =  Class Species Status);}
\item[Ex2] \ttt{DATA animals (DROP =  Habitat Sex);}
\item[Ex3] \ttt{MERGE animals1 animals2 (RENAME = (Sex=Gender));}
\ei
\end{frame}



\begin{frame}[fragile]
\ft{Discussion}
\bmp{.50\textwidth}
\footnotesize
\begin{craw}{.1}{Data Set MEN}
ID   NAME     GRADE
1    Andrew   B+
3    Jimmy    B-
5    Ulric    A-
\textcolor{OrangeRed}{8    Allan    B}
\end{craw}
%\vspace{.0in}
\begin{craw}{.1}{Data Set HEIGHT\_M}
ID HEIGHT
1  68
3  69
5  72
\textcolor{OrangeRed}{10 70}
\end{craw}
\vspace{.0in}
\emp
\bmp{.50\textwidth}
\begin{code}{.}
DATA merged;
   MERGE men height\_m;
   BY ID;
RUN;
\end{code}
%\vspace{.0in}
\begin{clicker}{How many \emph{observations} will be in the resulting data set?}
\begin{itemize}
\item[0.] none - there will be an error
\item[3.] 3
\item[4.] 4
\item[5.] 5
\end{itemize}
\end{clicker}
\emp
\vspace{.0in}
\end{frame}



\begin{frame}[fragile]
\ft{One to many merge}
\bi
\item[Goal:] combine data sets that have different numbers of observations
\item[How:] use the \ttt{MERGE} statement \ttt{BY} identifying variables (data must be pre-sorted)
\ei
\bmp{0.36\textwidth}
\footnotesize
\begin{craw}{.0}{Data Set PROF}
ID   NAME
1    Sklar
3    Doi
4    Peck
\end{craw}

\begin{code}{.0}
DATA merged;
   \textcolor{OrangeRed}{MERGE} prof class;
   \textcolor{OrangeRed}{BY} id;
RUN;
\end{code}
%\vspace{.0in}
\emp
\bmp{0.05\textwidth} \hspace{0.05in} \emp
\bmp{.27\textwidth}
\begin{craw}{.0}{Data Set CLASS}
ID CLASS
1  Stat218
1  Stat417
3  Stat150
3  Stat330
3  Stat418
4  Stat251
4  Stat323
4  Stat423
\end{craw}
\vskip25pt
\textcolor{White}{x}
\emp
\bmp{0.05\textwidth} \hspace{0.05in} \emp
\bmp{.35\textwidth}
%\vspace{.0in}
\begin{craw}{.0}{Data Set MERGED}
ID  NAME   CLASS
1   Sklar  Stat218
1   Sklar  Stat417
3   Doi    Stat150
3   Doi    Stat330
3   Doi    Stat418
4   Peck   Stat251
4   Peck   Stat323
4   Peck   Stat423
\end{craw}
\vskip25pt
\textcolor{White}{x}
\emp
\end{frame}


%===========================================================================================================================
\section[Tracking Observations]{Tracking Observations}
%===========================================================================================================================
\subsection{}
\begin{frame}
\tableofcontents[currentsection, hideallsubsections]
\end{frame}

\begin{frame}
\ft{Tracking with \ttt{IN=}}
\bi
\item When combining data sets, we can track if an individual observation is present/absent in only one data set or in both
\item The \fbox{\ttt{IN=}\emph{new-var-name}}  option creates a \emph{temporary} indicator variable with values of 0/1
\bi
\item[0 =] observation \underline{not} found in that data set
\item[1 =] observation found in that data set
\ei
\item Can be used with the \ttt{SET} or \ttt{MERGE} statements, but typically it is used with \ttt{MERGE}
\item These indicator variables are typically used for subsetting data
\item Visualize this with Venn diagrams:
\item[] \url{http://analisisydecision.es/wp-content/uploads/2014/12/tipos-de-merge-en-SAS.png}
\ei
\end{frame}


\begin{frame}[fragile]
\ft{Example}
\bmp{0.30\textwidth}
\footnotesize
\begin{craw}{.0}{Data Set MEMBERS}
ID  STATE
101 NC
102 CA
103 CA
104 WI
105 NY
\end{craw}
\footnotesize
\begin{craw}{.0}{Data Set ORDERS}
ID  TOTAL
102 30.01
104 254.98
104 75.00
101 1600.56
102 385.30
\end{craw}
\emp
\bmp{0.03\textwidth} \hspace{0.05in} \emp
\bmp{0.65\textwidth}
\bi
\item iFixit is a local SLO based company
\bi
\item provides \emph{free} repair guides (phones, washing machines, etc.)
\item makes money through selling repair tools and parts
\ei
\item One database stores member information, another stores member orders
\item Goal: identify members who haven't made a recent purchase
\ei
\oyo What is the data we want?
\vskip20pt
\textcolor{White}{x}
\emp
\end{frame}

\begin{frame}[fragile]
\ft{Discussion}
\bmp{0.45\textwidth}
\footnotesize
\begin{code}{.0}
DATA example;
   MERGE members (IN=a)
         orders  (IN=b);
    BY id;
    \textcolor{OrangeRed}{IF \emph{condition};}
RUN;
\end{code}
\emp
\bmp{0.05\textwidth} \hspace{0.05in} \emp
\bmp{0.45\textwidth}
\begin{clicker}{Which \ttt{if} statement should you use keep members who haven't made a recent purchase?}
\begin{enumerate}
\item \ttt{if a;}
\item \ttt{if b;}
\item \ttt{if a and b;}
\item \ttt{if a or b;}
\item \ttt{if a and not b;}
\item \ttt{if not a and b;}
\item \ttt{if not (a and b);}
\end{enumerate}
\end{clicker}

\emp
\end{frame}

\end{document} 